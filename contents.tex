\pagestyle{fancy}
\pagenumbering{arabic}

\section{Topic and Introduction}
\label{sec:Topic_and_Introduction}
In manufacturing companies, the proper assembly line plays a crucial role in ensuring efficient production processes and meeting customer demand. Nevertheless, with wrong assembly line configurations, companies may face challenges such as high production costs, low productivity, and delays. In the shopfloor of a manufacturing facility, assembly line balancing is therefore a crucial process that aims to optimise the allocation of tasks among workers and machines (Friedli et al., 2010). This process ensures that the assembly line operates efficiently and effectively, maximising productivity while minimising bottlenecks and idle time. There are several models in order to optimise and balance assembly lines, including the mixed model line balancing approach (Yusuf et al., 2020). This approach aims to balance the workload and maximise efficiency in assembly lines where multiple product models are being produced. The mixed model line balancing approach takes into consideration various factors such as workstation requirements, cycle times, and demand variability to determine the optimal allocation of tasks across workstations in order to prevent idle time. To address this challenge, we developed a software tool using Python programming language to faciliate the process of assembly line balancing for mixed model production. 

\section{Project Management}
In order to obtain a proper structure to fulfil the project requirements, we developed the known project management tools studied in the previous semesters. Those tools helped us to dissect different tasks and distribute them.

\subsection{Magic Triangle}
